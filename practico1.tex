\documentclass[12pt]{article}
\usepackage[utf8]{inputenc}
\usepackage[total={18cm,24cm},centering]{geometry} %define tamaño para imprimir hoja
\usepackage{tikz}
\usepackage{graphicx}
\usepackage{amssymb}
\usepackage{amsmath}
\usepackage{mathrsfs}
\usepackage{tcolorbox}
\tcbuselibrary{theorems}
\usepackage{color}
\usepackage[europeanresistors, americaninductors]{circuitikz}
\usetikzlibrary{calc}
\usetikzlibrary{patterns}


\usetikzlibrary{patterns,angles,quotes}

\ctikzset{bipoles/thickness=1} %grosor elementos pasivos dos polos
%\ctikzset{bipoles/length=1.2cm} %longitud de elementos pasivos dos polos
%\tikzstyle{every node}=[font\normalsize] %Tamaño de las etiquetas
\tikzstyle{every path}=[line width=1pt, line cap=round, line join=round] %Caracteristicas linea union


%\title{Apuntes de clase}
%\author{Lima Soto Ariel Wilson}
%\date{\today}



\begin{document}


\section{Ejercicio \# 1}

Hallar ecuaci\'on diferencial: \(\displaystyle FT_{1}(s)=\frac{X(s)}{R(s)}\) \hspace{5mm} y \hspace{5mm} \(\displaystyle FT_{2}(s)=\frac{Y(s)}{R(s)}\).

  %Diagrama masa, resorte y amortiguamiento
  \begin{circuitikz}
    %ground
    \fill[pattern=north east lines] (0,5) rectangle (8,5.25);
    \draw[thick] (0,5)--(8,5);

    \draw (2,5) to[spring, l=$\mathbf{k2}$] (2,2); %resorte
    \draw (6,5) to[damper, l_=$\mathbf{\beta}$] (6,2); %amortiguador
    \draw[fill=gray!40] (1,1) rectangle (7,2); %rectangulo
    \node at (4,1.5) {$m2$}; %nodo masa

    \draw[thick,->] (4,3.5) -- (4,2); %flecha
    \node at (4,3.75) {$\mathbf{r(t)}$};

    \draw (4,1) to[spring, l=$\mathbf{k1}$] (4,-2); %resorte
    \draw[fill=gray!40] (1,-2) rectangle (7,-3); %rectangulo
    \node at (4,-2.5) {$m1$}; %nodo masa

    \draw[thick,->] (7.5,1) -- (7.5,3);
    \draw[thick] (7.2,2) -- (8,2);
    \node at (7.5,3.3) {$x(t)$}; %nodo x

    \draw[thick,->] (7.5,-3) -- (7.5,-1);
    \draw[thick] (7.2,-2) -- (8,-2);
    \node at (7.5,-0.7) {$y(t)$}; %nodo y

    %Diagrama de fuerzas para m2
    \draw[thick] (10,4) -- (14,4);
    \draw[thick,->] (12,6) -- (12,4);
    \draw[thick,->] (10.5,2) -- (10.5,4);
    \draw[thick,->] (12,2) -- (12,4);
    \draw[thick,->] (13.5,2) -- (13.5,4);
    \node at (12,6.2) {$\mathbf{r(t)}$};
    \node at (10.5,1.8) {$\mathbf{F_{m}(t)}$};
    \node at (12,1.8) {$\mathbf{F_{\beta}(t)}$};
    \node at (13.5,1.8) {$\mathbf{F_{k}(t)}$};

    %Diagrama de fuerzas para m1
    \draw[thick] (10,-2) -- (14,-2);


\end{circuitikz}

\newpage

\section{Ejercicio \# 2}

Hallar: \(\displaystyle \frac{V_{o}(s)}{V_{i}(s)}\)

\begin{figure}[h]
%Diagrama de circuito RLC
\begin{circuitikz}[american]

    %\draw (0,0) to [sV=$v_{i}(t)$] ++(0,3);
    \draw (0,3) to [R=$R$] (3,3);
    \draw (3,3) to [L, l=$L$] (6,3);
    \draw (6,3) to [short, -*] (7.5,3); %nodo positivo
    \draw (6,3) to [C, l=$C$] (6,0);
    \draw (6,0) to [short, -*] (7.5,0); %nodo negativo
    \draw (7.5,3) to [open, v=$v_{0}(t)$] (7.5,0);
    \draw (6,0) to [short] (0,0)
          (0,0) to [sV, l=$v_{i}(t)$] (0,3);

\end{circuitikz}
\end{figure}

aAplicamos la ley de Kirchhoff de las mallas:

$$v_{i}(t) = v_{R}(t) + v_{L}(t) + v_{C}(t)$$

Donde:
$$v_{C}(t) = \frac{1}{C}\int i_{C}(t) \mathrm{d}t$$
$$v_{L}(t) = L\frac{\mathrm{d}i(t)}{\mathrm{d}t}$$
$$v_{R}(t) = Ri(t)$$

Ecuación diferencial del circuito RLC:
\begin{equation*}
  v_{i}(t) = Ri(t) + L\frac{\mathrm{d}i(t)}{\mathrm{d}t} + \frac{1}{C}\int i_{C}(t) \mathrm{d}t
\end{equation*}

Aplicamos la transformada de Laplace:
\begin{eqnarray*}
  \mathscr{L}\{v_{i}(t)\} = \mathscr{L}\{v_{R}(t) + v_{L}(t) + v_{C}(t)\} \\  [3mm]
  \mathscr{L}\{v_{i}(t)\} = \mathscr{L}\left \{Ri(t) + L\frac{\mathrm{d}i(t)}{\mathrm{d}t} + \frac{1}{C}\int i_{C}(t) \mathrm{d}t\right \} \\ [3mm]
  V_{i}(s) = RI(s) + LSI(s) + \frac{1}{CS}I(s) \\ [3mm]
  V_{i}(s) = I(s)\left (R + LS + \frac{1}{CS} \right) \\ [3mm]
  V_{i}(s) = I(s)\left (\frac{S^{2}LC+1+SRC}{SC} \right) \\ [3mm]
  I(s) = V_{i}(s)\left (\frac{SC}{S^{2}LC+1+SRC} \right) \\ [3mm]
\end{eqnarray*}

Aplicamos la Ley de tension de Kirchhoff:
\begin{eqnarray*}
  -v_{o}(t) + v_{c}(t) = 0 \\ [3mm]
  v_{o}(t) = v_{C}(t) = \frac{1}{C}\int i_{C}(t) \mathrm{d}t \\ [3mm]
  \mathscr{L}\{v_{o}(t)\} = \mathscr{L}\left \{\frac{1}{C}\int i_{C}(t) \mathrm{d}t\right \} \\ [3mm]
  V_{o}(s) = \frac{1}{CS}I(s) \\ [3mm]
\end{eqnarray*}

Luego reemplazamos:
\begin{eqnarray*}
  V_{o}(s) = \frac{1}{SC}\left[ V_{i}(s)\left( \frac{SC}{S^{2}LC+1+SRC} \right) \right] \\ [3mm]
  V_{o}(s) = V_{i}(s)\left( \frac{1}{S^{2}LC+1+SRC} \right) \\ [3mm]
\end{eqnarray*}

Solución:
\begin{eqnarray*}
\tcboxmath[colback=blue!25,colframe=black] {\frac{V_{o}(s)}{V_{i}(s)} = \frac{1}{S^{2}LC+1+SRC}}
\end{eqnarray*}

\newpage

\section{Ejercicio \# 3}
%Ejercicio 3 Disgrama de bloque de sistemas de control.



\end{document}
